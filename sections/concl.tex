To reuse:

Domain graphs adopt the natural idea of adding edge ids to directed labeled edges in order to concisely model higher-arity relations in graphs, as needed in Wikidata, without the need for reserved vocabulary or reification. They can naturally represent popular graph models, such as RDF and property graphs, and allow for combining the features of both models in a novel way. While the idea of using edge ids as a hook for modeling higher-arity relations in graphs is far from new (see, e.g.,~\cite{HernandezHK15,IlievskiGCDYRLL20,LassilaSBBBKKLST}), it is an idea that is garnering increased attention as a more flexible and concise alternative to reification.
%, RDF*, property graphs, etc.
To the best of our knowledge, our work is the first to propose a formal data model that incorporates edge ids, a query language that can take advantage of them, and a fully-fledged graph database engine that supports them by design. We also propose to optionally allow (external) annotations on top of the graph structure, thus facilitating better compatibility with property graphs, whereby labels and property--values can be added to graph objects without adding new nodes and edges to the graph~itself.

