\newcommand{\objs}{\mathsf{Obj}}
\newcommand{\labels}{\mathcal{L}}
\newcommand{\keys}{\mathcal{K}}
\newcommand{\cprop}{\mathcal{P}}
\newcommand{\values}{\mathcal{V}}
\newcommand{\vars}{\mathsf{Var}}
\newcommand{\pvars}{\mathsf{PathVar}}


\def\ojoin{\setbox0=\hbox{$\Join$}\rule[0.1ex]{.25em}{.6pt}\llap{\rule[1ex]{.25em}{.6pt}}}
\def\leftouterjoin{\mathbin{\ojoin\mkern-6.6mu\Join}}
\newcommand{\LOJ}{\mathbin{\leftouterjoin}}


%\newcommand{\dom}{\text{dom}}
\newcommand{\Dom}[1]{\mathsf{dom}(#1)}
\newcommand{\Var}[1]{\mathsf{var}(#1)}
\newcommand{\inv}[1]{#1^{-}}


\newcommand{\I}{\text{\bf I}}
\newcommand{\Lit}{{\bf L}}
\newcommand{\V}{{\bf V}}
\newcommand{\B}{{\bf B}}
\newcommand{\N}{{\bf N}}

\newcommand{\map}{\mu}
\newcommand{\mset}{M}
\newcommand{\mappings}{\mathcal{M}}
\newcommand{\sem}[1]{\llbracket #1\rrbracket}
\newcommand{\semp}[2]{\llbracket #1\rrbracket_{#2}}
\newcommand{\sempdg}[3]{\llbracket #1\rrbracket^{#3}_{#2}}
\newcommand{\sempd}[2]{\llbracket #1\rrbracket^{#2}}
\newcommand{\eval}[2]{\mathsf{Eval}(#1,#2)}


\newcommand{\avar}[1]{\mathsf{AVars}(#1)}
\newcommand{\evar}[1]{\mathsf{EVars}(#1)}
\newcommand{\qvars}[1]{\mathsf{Vars}(#1)}

\newcommand{\lab}{\texttt{lab}}
\newcommand{\prop}{\texttt{prop}}
\newcommand{\simc}{\ensuremath{\mathfrak{s}}}
\DeclarePairedDelimiter{\ceil}{\lceil}{\rceil}

\newcommand{\dle}[3]{\ensuremath{\texttt{#1} \xrightarrow{\texttt{#2}} \texttt{#3}}}


\lstdefinestyle{ttl}
{	language=SPARQL, 
	basicstyle=\ttfamily\scriptsize, 
	captionpos=b, 
	%frame=single, 
	showstringspaces=false, 
	numbers=none, 
	morekeywords={SIMILARITY, TOP, MINUS, WITHIN, SERVICE, PREFIX, SELF, SELECT, MATCH, WHERE, FILTER}
}

\lstset{style=ttl}

%\newcommand{\ah}[1]{{\color{blue}\textsc{aidan}: #1}}
%\newcommand{\dv}[1]{{\color{green}\textsc{domagoj}: #1}}

%Better command environment :)
\newcommand{\dv}[1]{\todo[inline, color=green!30]{Domagoj: #1}}
\newcommand{\ma}[1]{\todo[inline, color=orange!40]{Marcelo: #1}}
\newcommand{\ah}[1]{\todo[inline, color=blue!20]{Aidan: #1}}
\newcommand{\ra}[1]{\todo[inline, color=yellow!40]{Renzo: #1}}
\newcommand{\riveros}[1]{\todo[inline, color=red!30]{\textbf{Riveros}: #1}}

\makeatletter
\newenvironment{customlegend}[1][]{%
    \begingroup
    % inits/clears the lists (which might be populated from previous
    % axes):
    \pgfplots@init@cleared@structures
    \pgfplotsset{#1}%
}{%
    % draws the legend:
    \pgfplots@createlegend
    \endgroup
}%

\def\addlegendimage{\pgfplots@addlegendimage}
\makeatother

\newcommand{\AS}{\operatorname{AS}}
\newcommand{\AAND}{~\operatorname{AND}~}
\newcommand{\OPT}{~\operatorname{OPT}~}
\newcommand{\FILTER}{~\operatorname{FILTER}~}
\newcommand{\SELECT}{\operatorname{SELECT}~}
\newcommand{\tru}{\operatorname{true}}
\newcommand{\fal}{\operatorname{false}}
\newcommand{\err}{\operatorname{error}}


\lstdefinelanguage{dql}{
    keywords = {SELECT,MATCH,WHERE,OPTIONAL,TYPE,AND,OR,ASC,DESC,LIMIT,ORDER,BY}
}

\lstdefinestyle{dqls}{
    inputencoding=utf8,
    basicstyle=\ttfamily\small,
    language=dql,
    %columns=fullflexible,
    basewidth=0.52em, % keep monospace for bold
    keywordstyle=\textbf,
    upquote=true,
    escapechar=`
}


\newcommand{\dqlt}[1]{\texttt{#1}}
\newcommand{\dqlkw}[1]{\textbf{\dqlt{#1}}}

\lstnewenvironment{dql}[1][]{%
    \noindent\minipage[b]{\linewidth}%
    \centering%
    \tabular{@{}c@{}}%
    \lstset{style=dqls,#1}
}{\endtabular\endminipage\vspace{5pt}}



\usetikzlibrary{arrows,positioning,backgrounds} 
\tikzset{
%    rt/.style={
%		rectangle,
%		fill = white,
%		draw=black, 
%		text centered,
%		inner sep=0.5ex
%		},
%    rtt/.style={ %tighter version
%    	rt,
%    	inner sep=0.1ex
%    	},
    ert/.style={ %edge box
     	rt,
     	dashed
     	}, 
%    ertt/.style={ %edge box, tighter
%        rtt,
%        dashed
%        }, 
%    rect/.style={ % rounded prop graph boxes
%        rectangle,
%        fill = white,
%        rounded corners,
%        draw=black, 
%        text centered,
%        inner sep=0.8ex
%        },
%    rectw/.style={
%        rect,
%        draw=white
%        },
%    erect/.style={ %edge rect, unfortunate name
%    	rect,
%    	dashed
%    	},
    erectw/.style={ %edge rectw
     	rectw,
     	dashed
     	},
%    arrout/.style={
%           ->,
%           -latex,
%           },
%    arrin/.style={
%           <-,
%           latex-,
%           },
%    arrb/.style={
%           <->,
%           >=latex,
%           }
}

%\newcommand{\ym}{\ding{51}}%
%\newcommand{\nm}{\ding{55}}%
