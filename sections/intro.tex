Recent years have seen renewed interest in using graphs for modelling, managing, querying and analysing data, particularly in scenarios involving diverse data, incomplete knowledge, multitudinous sources, and so forth. This interest stems from the growing realisation in various communities -- such as Databases~\cite{Bonifati}, Semantic Web~\cite{Hitzler2010}, Machine Learning~\cite{Hamilton}, and more recently Knowledge Graphs~\cite{HoganBCdMGKGNNN21} -- that graphs provide a flexible, lightweight and intuitive abstraction well-suited to many complex, diverse domains~\cite{AnglesABBFGLPPS18}.

Within these different communities, a variety of both abstract and concrete graph-based data models have been proposed.\footnote{By an \textit{abstract data model}, herein we refer to a structure used to represent data (e.g., the relational model). A \textit{concrete data model} further adds details important in practice, such as the types of terms allowed, syntaxes, etc. (e.g., SQL's data model).} Perhaps the simplest such model is that of a directed labelled graph, which is simply a set of triples, where each triple forms a directed labelled edge. This forms the basis of concrete data models, such as the Resource Description Framework~\cite{CyganiakWL14} proposed within the Semantic Web community. Such an abstraction is now also popular in the Machine Learning community, forming the basis of topics such as knowledge graph embeddings~\cite{Wang2017KGEmbedding}. It is also popular in the Database community, where such graphs are often simply called \textit{graph databases}, particularly in the more theoretical literature~\cite{Wood12}.

However, in practice, directed labelled graphs are sometimes considered \textit{too simple}. What if, for example, we want to add data that describe edges themselves, or graphs themselves? While more complex data can be modelled in directed labelled graphs using various forms of reification~\cite{HernandezHK15}, the result can often be verbose and unintuitive~\cite{AnglesABBFGLPPS18}. Hence a wide range of graph-based models have emerged down through the years~\cite{AnglesG08}. More recently, the (labelled) property graph model~\cite{Webber12,AnglesABBFGLPPS18} has gained significant popularity in the Database community, while models such as named graphs~\cite{HarrisS13} and RDF-star~\cite{Hartig17} have been proposed within the Semantic Web community as alternatives to reification. 

With several abstract and concrete graph models now available, the question becomes how to make these models \textit{interoperable}. How can we integrate data from both RDF and property graphs? How can we design a graph database engine that can seamlessly ingest, integrate and query data from any such model? How can we layer graph analytics or machine learning over such models? One possibility is to take the simplest model -- directed labelled graphs -- as our base and use reification~\cite{HernandezHK15} to represent more complex models, but as mentioned before, reification is too verbose. Another possibility is to take a more complex model -- say property graphs -- as our base~\cite{Hartig14,AnglesTT20}, but this would add complexity to higher levels when we think of graph queries, analytics, learning, etc.

Herein, we propose an intermediate solution. Specifically we propose \textit{one graph}: an abstract graph model that extends directed labelled graphs with edge ids~\cite{IlievskiGCDYRLL20,LassilaSBBBKKLST,VrgocRAAABHNRR21}. This model removes the need for reification when dealing with more complex models, and yet adds minimal complexity versus directed labelled graphs. 

We will first introduce existing graph models, and their strengths and weaknesses. We then introduce, motivate, and formally define the one graph model, and show how other graph models can be represented within it. We further discuss practical benefits of the model, and how it is currently being used. We conclude with some future directions regarding the model.