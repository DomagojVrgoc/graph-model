One of the simplest models used for representing knowledge graphs is based on \textit{directed labelled graphs}, which consist of a set of nodes and edges such that every edge relates a pair of nodes. An edge is graphically represented as \gedge[arrin][0.6cm]{$a$}{$b$}{$c$}, where $a$ is called the source node, $b$ the edge label, and $c$ the target node. Such graphs are a staple in the theoretical literature on graph databases \cite{Baeza13}, and they form the basis of the RDF data model~\cite{CyganiakWL14}, where the source node, edge label and target node are called \textit{subject}, \textit{predicate} and \textit{object}, respectively. In the context of knowledge graphs, nodes are used to represent entities and edges represent binary relations. As an example, the edge \gedge[arrin][1.8cm]{Michelle Bachelet}{position held}{President of Chile}, tells us that Michelle Bachelet was (or is) the president of Chile.
%A popular concrete model based on \textit{directed edge-labelled graphs} is the RDF model~\cite{CyganiakWL14}. 

However, directed labelled graphs are sometimes \textit{too} simple. While they elegantly represent binary relations, they are cumbersome when representing higher arity relations. Take, for example, the two statements from the Wikidata knowledge graph~\cite{VrandecicK14} illustrated in Figure~\ref{fig:mb}. Both statements claim that Michelle Bachelet was a president of Chile, and both are associated with nested \textit{qualifiers} that provide additional information: in this case a start date, an end date, who replaced her, and whom she was replaced by. There are two statements, indicating two distinct periods when she held the position. Also the ids for objects (for example, \textsf{Q320} and \textsf{P39}) are shown; any positional element can have an id and be viewed as a node in the knowledge graph (for instance \texttt{start date}, identified by \texttt{P580} can be a source node of another statement). 

Representing statements like this in a directed labelled graph requires some form of \textit{reification} to decompose $n$-ary relations into binary relations~\cite{HernandezHK15}. For example, Figure~\ref{fig:delg} shows a graph where $e_1$ and $e_2$ are nodes representing $n$-ary relationships. The reification is given by the use of the edges labelled as \texttt{source}, \texttt{label} and \texttt{target}. For simplicity, we use human-readable nodes and labels, where in practice, a node \gnode{Sebastián Piñera} will rather be given as the identifier \gnode{Q306}, and an edge label \texttt{replaces} will rather be given as \texttt{P155}. While using reification is a valid solution, its main drawbacks are that: (i) it can easily become cumbersome and inefficient for querying; and (ii) it introduces semantics into graph data (requiring that an edge labelled \texttt{replaces} has a particular meaning for instance).

\newcommand{\wid}[1]{{\color{gray}~[#1]}} % format Wikidata IDs

\begin{figure}[tb]
\centering
{\sf
\begin{tabular}{l}
\large\textbf{Michelle Bachelet\wid{Q320}}\\[1ex]
\large\qquad position held\wid{P39} \quad President of Chile\wid{Q466956}\\
\qquad \qquad 
{\begin{tabular}{l@{\qquad~~}l}
start date \wid{P580} & 2014-03-11\\
end date \wid{P582} 	& 2018-03-11\\
replaces\wid{P155}	& Sebastián Piñera\wid{Q306}\\
replaced by\wid{P156} & Sebastián Piñera\wid{Q306}\\
\end{tabular}}\\
\\[-1ex]
\large\qquad position held\wid{P39} \quad President of Chile\wid{Q466956}\\
\qquad \qquad {\begin{tabular}{l@{\qquad~~}l}
start date \wid{P580} & 2006-03-11\\
end date \wid{P582} & 2010-03-11\\
replaces\wid{P155} & Ricardo Lagos\wid{Q331}\\
replaced by\wid{P156} & Sebastián Piñera\wid{Q306}\\
\end{tabular}}
\end{tabular}
}
\caption{Wikidata statement group for Michelle Bachelet \label{fig:mb}}
\end{figure}


%\begin{figure}[t]
%\centering
%\includegraphics[width=8.5cm]{./images/digraph.png}
%\caption{Directed edge-labelled graph reifying the statements of Figure~\ref{fig:mb} \label{fig:delg}}
%\end{figure}


\begin{figure}[t]
\setlength{\vgap}{0.45cm}
\setlength{\hgap}{2.5cm}
\centering
\begin{tikzpicture}
\node[iri] (e1) {$e_1$};

\node[iri,right=\hgap of e1] (pe1) {position held}
  edge[arrin] node[lab] {label} (e1);
  
\node[iri,above=\vgap of pe1] (se1) {Michelle Bachelet}
  edge[arrin] node[lab] {source} (e1);
  
\node[iri,below=\vgap of pe1] (oe1) {President of Chile}
  edge[arrin] node[lab] {target} (e1);
 
\node[iri,above=\vgap of se1] (sp) {Sebastian Piñera}
  edge[arrin] node[lab] {replaced by} (e1)
  edge[arrin,bend right=19] node[lab,pos=0.45] {replaces} (e1);
  
\node[iri,right=\hgap of pe1] (e2) {$e_2$}
  edge[arrout] node[lab] {label} (pe1)
  edge[arrout] node[lab] {source} (se1)
  edge[arrout] node[lab] {target} (oe1)
  edge[arrout] node[lab] {replaced by} (sp);
  
\node[iri,below=2.2\vgap of e1] (sd1) {2014-03-11}
  edge[arrin] node[lab] {start date} (e1);
\node[iri,above=2.2\vgap of e1] (ed1) {2018-03-11}
  edge[arrin] node[lab] {end date} (e1);

\node[iri,below=2.2\vgap of e2] (sd2) {2006-03-11}
  edge[arrin] node[lab] {start date} (e2);
\node[iri,above=2.2\vgap of e2] (ed2) {2010-03-11}
  edge[arrin] node[lab] {end date} (e2);
  
\node[iri,below=\vgap of oe1] (rl) {Ricardo Lagos}
  edge[arrin,bend right=17] node[lab] {replaces} (e2);
\end{tikzpicture}
\caption{Directed edge-labelled graph reifying the statements of Figure~\ref{fig:mb} \label{fig:delg}}
\end{figure}



To circumvent this issue, a number of graph models have been proposed to capture higher-arity relations more concisely, including property graphs~\cite{FrancisGGLLMPRS18} and RDF*~\cite{Hartig17,Hartig21}. However, both have limitations that render them incapable of modeling the statements shown in Figure~\ref{fig:mb} without resorting to reification~\cite{HoganRRS19}. On the one hand, property graphs allow labels and property--value pairs to be associated with both nodes and edges. For example, the statements of Figure~\ref{fig:mb} can be represented as the property graph in Figure \ref{fig:pg}. 
Though more concise than reification, labels, properties and values are considered to be simple strings, which are disjoint with nodes; for example, \textsf{"Ricardo Lagos"} is neither a node nor a pointer to a node, but a string, which would complicate, for example, querying for the parties of presidents that Michelle Bachelet replaced. 

\begin{figure}
\setlength{\vgap}{1.5cm}
\setlength{\hgap}{1cm}
\centering
%\begin{center}
\vspace{2pt}
\setlength{\vgap}{1.5cm}
\setlength{\hgap}{1.5cm}
\begin{tikzpicture}
  %%% node n1 
  \node[nrect] (n1) {
   \alt{
       \uri{name} & =\,\uri{"Michelle Bachelet"} } };
  %%% label of n1
  \node[rt] (ln1) at (n1.north) {$n_1$ : \uri{human}};
  
  %%% node n2
  \node[nrect,right=\hgap of n1] (n2) {
   \alt{
      \uri{name} & =\,\uri{"President of Chile"} } };
  %%% label of n2
  \node[rt] (ln2) at (n2.north) {$n_2$ : \uri{public office}};

  %%% edge e1
  \draw[arrout,pos=0.5,bend left=25] (ln1) to node[rte,yshift=-0.1cm] (le1)
   {$e_1$ $:$ position held}
  (ln2);
  %%% atributes e1
  \node[erect,anchor=south,yshift=-0.5ex] at (le1.north) (e1) {
     \alt{ \uri{start date} & =\,\uri{"11 March 2014"}\\[-0.7em]
           \uri{end date} & =\,\uri{"11 March 2018"}\\[-0.7em]
           \uri{replaces} & =\,\uri{"Sebastián Piñera"}\\[-0.7em]
           \uri{replaced by} & =\,\uri{"Sebastián Piñera"} } };
  
  %%% edge e2
  \draw[arrout,pos=0.5,bend right=25] (n1) to node[rte,yshift=-0.1cm] (le2) {$e_2$ $:$ position held} (n2);
  %%% atributes e2  
  \node[erect,anchor=north,yshift=0.5ex] (e2) at (le2.south) {
     \alt{ \uri{start date} & =\,\uri{"11 March 2006"}\\[-0.7em]
           \uri{end date} & =\,\uri{"11 March 2010"}\\[-0.7em]
           \uri{replaces} & =\,\uri{"Ricardo Lagos"}\\[-0.7em]
           \uri{replaced by} & =\,\uri{"Sebastián Piñera"}\\ } };
\end{tikzpicture}
\vspace{2pt}
%\end{center}
\caption{Property graph for the statements of Figure~\ref{fig:mb}, with some additional data\label{fig:pg}}
\end{figure}




On the other hand, RDF* allows an edge to be a node. For example, the first statement of Figure~\ref{fig:mb} can be represented as follows in RDF* as shown in Figure \ref{fig:rdf*}. The node representing the edge is called a quoted triple~\cite{Hartig21}.
However, we can only represent one of the statements (without reification), as we can only have one distinct node per edge; if we add the qualifiers for both statements, then we would not know which start date pairs with which end date, for example. A proposed workaround involves adding intermediate nodes to denote different \textit{occurrences} of quoted triples, but this requires a reserved term~\cite{Hartig21}.
%Finally, RDF datasets model multiple named graphs, where we could define a graph with a single edge and define qualifiers on that graph's name~\cite{HernandezHK15}; however, named graphs are intended for tracking the source of a graph.


\begin{figure}[tb]
\setlength{\vgap}{0.8cm}
\setlength{\hgap}{1.9cm}
\centering
\begin{tikzpicture}
\node[iri,anchor=center,minimum width=6.3cm,minimum height=0.7cm,dashed] (e1) {};
\node[iri,right=1ex of e1.west,anchor=west] (sp1) {Michelle Bachelet};
\node[iri,right=\hgap of sp1] (ch1) {President of Chile}
  edge[arrin] node[iri,draw=none] {position held} (sp1);
  
\node[iri,below=\vgap of e1] (mb) {Sebastián Piñera}
  edge[arrin,bend right=40] node[lab,xshift=1ex] {replaces} (e1)
  edge[arrin,bend left=40] node[lab,xshift=-1ex] {replaced by} (e1);

\node[iri,left=\hgap of mb] (ttt) {2018-03-11}
  edge[arrin] node[lab] {start date} (e1);

\node[iri,right=\hgap of mb] (ttf) {2014-03-11}
  edge[arrin] node[lab] {end date} (e1);
\end{tikzpicture}
\caption{RDF* graph for the first statement of Figure~\ref{fig:mb} \label{fig:rdf*}}
\end{figure}

