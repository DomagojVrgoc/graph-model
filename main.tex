%%
%% This is file `sample-sigconf.tex',
%% generated with the docstrip utility.
%%
%% The original source files were:
%%
%% samples.dtx  (with options: `sigconf')
%% 
%% IMPORTANT NOTICE:
%% 
%% For the copyright see the source file.
%% 
%% Any modified versions of this file must be renamed
%% with new filenames distinct from sample-sigconf.tex.
%% 
%% For distribution of the original source see the terms
%% for copying and modification in the file samples.dtx.
%% 
%% This generated file may be distributed as long as the
%% original source files, as listed above, are part of the
%% same distribution. (The sources need not necessarily be
%% in the same archive or directory.)
%%
%% The first command in your LaTeX source must be the \documentclass command.
\documentclass[sigconf]{acmart}

\settopmatter{printacmref=false} % Removes citation information below abstract
\renewcommand\footnotetextcopyrightpermission[1]{} % removes footnote with conference information in first column
\pagestyle{plain} % removes running headers

%%
%% \BibTeX command to typeset BibTeX logo in the docs
\AtBeginDocument{%
  \providecommand\BibTeX{{%
    \normalfont B\kern-0.5em{\scshape i\kern-0.25em b}\kern-0.8em\TeX}}}

%% Rights management information.  This information is sent to you
%% when you complete the rights form.  These commands have SAMPLE
%% values in them; it is your responsibility as an author to replace
%% the commands and values with those provided to you when you
%% complete the rights form.
%\setcopyright{acmcopyright}
%\copyrightyear{2018}
%\acmYear{2018}
%\acmDOI{10.1145/1122445.1122456}
%
%%% These commands are for a PROCEEDINGS abstract or paper.
%\acmConference[Woodstock '18]{Woodstock '18: ACM Symposium on Neural
%  Gaze Detection}{June 03--05, 2018}{Woodstock, NY}
%\acmBooktitle{Woodstock '18: ACM Symposium on Neural Gaze Detection,
%  June 03--05, 2018, Woodstock, NY}
%\acmPrice{15.00}
%\acmISBN{978-1-4503-XXXX-X/18/06}
\usepackage{enumitem}

% For drawing graphs
\usepackage{kg-macros}
\usepackage{mathtools}
\usepackage{listings} 

%For comments
\usepackage{todonotes}

%For plots
\usepackage{pgfplots}

\usepackage{booktabs}

% Generic names:
\newcommand{\ddata}{domain graph\xspace}
\newcommand{\ddatas}{domain graphs\xspace}
\newcommand{\DData}{Domain graph\xspace}
\newcommand{\DDatas}{Domain graphs\xspace}
\newcommand{\data}{one graph\xspace}
\newcommand{\datas}{one graphs\xspace}
\newcommand{\Data}{One graph\xspace}
\newcommand{\Datas}{One graphs\xspace}

\newcommand{\ql}{DGQL\xspace}

\usepackage{hyperref}
\hypersetup{
	colorlinks, linkcolor={black},
	citecolor={black}, urlcolor={black},
	pdftitle={1G: A unifying data model for graph databases},    % title
	pdfauthor={...}     % author
}

\newcommand{\objs}{\mathsf{Obj}}
\newcommand{\labels}{\mathcal{L}}
\newcommand{\keys}{\mathcal{K}}
\newcommand{\cprop}{\mathcal{P}}
\newcommand{\values}{\mathcal{V}}
\newcommand{\vars}{\mathsf{Var}}
\newcommand{\pvars}{\mathsf{PathVar}}


\def\ojoin{\setbox0=\hbox{$\Join$}\rule[0.1ex]{.25em}{.6pt}\llap{\rule[1ex]{.25em}{.6pt}}}
\def\leftouterjoin{\mathbin{\ojoin\mkern-6.6mu\Join}}
\newcommand{\LOJ}{\mathbin{\leftouterjoin}}


%\newcommand{\dom}{\text{dom}}
\newcommand{\Dom}[1]{\mathsf{dom}(#1)}
\newcommand{\Var}[1]{\mathsf{var}(#1)}
\newcommand{\inv}[1]{#1^{-}}


\newcommand{\I}{\text{\bf I}}
\newcommand{\Lit}{{\bf L}}
\newcommand{\V}{{\bf V}}
\newcommand{\B}{{\bf B}}
\newcommand{\N}{{\bf N}}

\newcommand{\map}{\mu}
\newcommand{\mset}{M}
\newcommand{\mappings}{\mathcal{M}}
\newcommand{\sem}[1]{\llbracket #1\rrbracket}
\newcommand{\semp}[2]{\llbracket #1\rrbracket_{#2}}
\newcommand{\sempdg}[3]{\llbracket #1\rrbracket^{#3}_{#2}}
\newcommand{\sempd}[2]{\llbracket #1\rrbracket^{#2}}
\newcommand{\eval}[2]{\mathsf{Eval}(#1,#2)}


\newcommand{\avar}[1]{\mathsf{AVars}(#1)}
\newcommand{\evar}[1]{\mathsf{EVars}(#1)}
\newcommand{\qvars}[1]{\mathsf{Vars}(#1)}

\newcommand{\lab}{\texttt{lab}}
\newcommand{\prop}{\texttt{prop}}
\newcommand{\simc}{\ensuremath{\mathfrak{s}}}
\DeclarePairedDelimiter{\ceil}{\lceil}{\rceil}

\newcommand{\dle}[3]{\ensuremath{\texttt{#1} \xrightarrow{\texttt{#2}} \texttt{#3}}}


\lstdefinestyle{ttl}
{	language=SPARQL, 
	basicstyle=\ttfamily\scriptsize, 
	captionpos=b, 
	%frame=single, 
	showstringspaces=false, 
	numbers=none, 
	morekeywords={SIMILARITY, TOP, MINUS, WITHIN, SERVICE, PREFIX, SELF, SELECT, MATCH, WHERE, FILTER}
}

\lstset{style=ttl}

%\newcommand{\ah}[1]{{\color{blue}\textsc{aidan}: #1}}
%\newcommand{\dv}[1]{{\color{green}\textsc{domagoj}: #1}}

%Better command environment :)
\newcommand{\dv}[1]{\todo[inline, color=green!30]{Domagoj: #1}}
\newcommand{\ma}[1]{\todo[inline, color=orange!40]{Marcelo: #1}}
\newcommand{\ah}[1]{\todo[inline, color=blue!20]{Aidan: #1}}
\newcommand{\ra}[1]{\todo[inline, color=yellow!40]{Renzo: #1}}
\newcommand{\riveros}[1]{\todo[inline, color=red!30]{\textbf{Riveros}: #1}}

\makeatletter
\newenvironment{customlegend}[1][]{%
    \begingroup
    % inits/clears the lists (which might be populated from previous
    % axes):
    \pgfplots@init@cleared@structures
    \pgfplotsset{#1}%
}{%
    % draws the legend:
    \pgfplots@createlegend
    \endgroup
}%

\def\addlegendimage{\pgfplots@addlegendimage}
\makeatother

\newcommand{\AS}{\operatorname{AS}}
\newcommand{\AAND}{~\operatorname{AND}~}
\newcommand{\OPT}{~\operatorname{OPT}~}
\newcommand{\FILTER}{~\operatorname{FILTER}~}
\newcommand{\SELECT}{\operatorname{SELECT}~}
\newcommand{\tru}{\operatorname{true}}
\newcommand{\fal}{\operatorname{false}}
\newcommand{\err}{\operatorname{error}}


\lstdefinelanguage{dql}{
    keywords = {SELECT,MATCH,WHERE,OPTIONAL,TYPE,AND,OR,ASC,DESC,LIMIT,ORDER,BY}
}

\lstdefinestyle{dqls}{
    inputencoding=utf8,
    basicstyle=\ttfamily\small,
    language=dql,
    %columns=fullflexible,
    basewidth=0.52em, % keep monospace for bold
    keywordstyle=\textbf,
    upquote=true,
    escapechar=`
}


\newcommand{\dqlt}[1]{\texttt{#1}}
\newcommand{\dqlkw}[1]{\textbf{\dqlt{#1}}}

\lstnewenvironment{dql}[1][]{%
    \noindent\minipage[b]{\linewidth}%
    \centering%
    \tabular{@{}c@{}}%
    \lstset{style=dqls,#1}
}{\endtabular\endminipage\vspace{5pt}}



\usetikzlibrary{arrows,positioning,backgrounds} 
\tikzset{
%    rt/.style={
%		rectangle,
%		fill = white,
%		draw=black, 
%		text centered,
%		inner sep=0.5ex
%		},
%    rtt/.style={ %tighter version
%    	rt,
%    	inner sep=0.1ex
%    	},
    ert/.style={ %edge box
     	rt,
     	dashed
     	}, 
%    ertt/.style={ %edge box, tighter
%        rtt,
%        dashed
%        }, 
%    rect/.style={ % rounded prop graph boxes
%        rectangle,
%        fill = white,
%        rounded corners,
%        draw=black, 
%        text centered,
%        inner sep=0.8ex
%        },
%    rectw/.style={
%        rect,
%        draw=white
%        },
%    erect/.style={ %edge rect, unfortunate name
%    	rect,
%    	dashed
%    	},
    erectw/.style={ %edge rectw
     	rectw,
     	dashed
     	},
%    arrout/.style={
%           ->,
%           -latex,
%           },
%    arrin/.style={
%           <-,
%           latex-,
%           },
%    arrb/.style={
%           <->,
%           >=latex,
%           }
}

%\newcommand{\ym}{\ding{51}}%
%\newcommand{\nm}{\ding{55}}%



%%
%% Submission ID.
%% Use this when submitting an article to a sponsored event. You'll
%% receive a unique submission ID from the organizers
%% of the event, and this ID should be used as the parameter to this command.
%%\acmSubmissionID{123-A56-BU3}

%%
%% The majority of ACM publications use numbered citations and
%% references.  The command \citestyle{authoryear} switches to the
%% "author year" style.
%%
%% If you are preparing content for an event
%% sponsored by ACM SIGGRAPH, you must use the "author year" style of
%% citations and references.
%% Uncommenting
%% the next command will enable that style.
%%\citestyle{acmauthoryear}

%%
%% end of the preamble, start of the body of the document source.
\begin{document}

\title[One graph: A unifying data model for graph databases]{One Graph: A unifying data model for graph databases}
%%
%% The "author" command and its associated commands are used to define
%% the authors and their affiliations.
%% Of note is the shared affiliation of the first two authors, and the
%% "authornote" and "authornotemark" commands
%% used to denote shared contribution to the research.
%\author{IMFD team}
%\affiliation{%
%  \institution{IMFD Chile}
%%  \streetaddress{P.O. Box 1212}
%%  \city{Dublin}
%%  \state{Ohio}
%  %  \postcode{43017-6221}
%  \country{Chile}
%}
%\email{imfd@imfd.cl}


\author[A. Hogan]{Aidan Hogan}
\affiliation{%
  \institution{DCC, University of Chile \& IMFD}
  \country{Chile}
}
\email{ahogan@dcc.uchile.cl}

\author[C. Rojas]{Carlos Rojas}
\affiliation{%
  \institution{IMFD}
  \country{Chile}
}
\email{cirojas6@uc.cl}

\author[D. Vrgo\v{c}]{Domagoj Vrgo\v{c}}
\affiliation{%
  \institution{PUC Chile \& IMFD}
  \country{Chile}
}
\email{dvrgoc@ing.puc.cl}


%%
%% By default, the full list of authors will be used in the page
%% headers. Often, this list is too long, and will overlap
%% other information printed in the page headers. This command allows
%% the author to define a more concise list
%% of authors' names for this purpose.
%\renewcommand{\shortauthors}{Trovato and Tobin, et al.}
%\renewcommand{\shortauthors}{Vrgo\v{c} and Rojas, et al.}

%%
%% The abstract is a short summary of the work to be presented in the
%% article.
\begin{abstract}
In this position paper, we argue that there is a need for a unifying  data model which can support popular graph formats such as RDF, RDF* and Property Graphs, while at the same time being powerful enough to naturally store information from complex knowledge graphs, such as WikiData, without the need for a complex reification scheme. Our proposal, called One Graph, or 1G for short, presents a simple and expressive data model for graphs which can naturally support all of the above, and more. We also observe that the idea of One Graph has appeared in existing graph systems from different vendors and research groups, illustrating its versatility in different application scenarios.
\end{abstract}

%%
%% The code below is generated by the tool at http://dl.acm.org/ccs.cfm.
%% Please copy and paste the code instead of the example below.
%%%
%\begin{CCSXML}
%<ccs2012>
% <concept>
%  <concept_id>10010520.10010553.10010562</concept_id>
%  <concept_desc>Computer systems organization~Embedded systems</concept_desc>
%  <concept_significance>500</concept_significance>
% </concept>
% <concept>
%  <concept_id>10010520.10010575.10010755</concept_id>
%  <concept_desc>Computer systems organization~Redundancy</concept_desc>
%  <concept_significance>300</concept_significance>
% </concept>
% <concept>
%  <concept_id>10010520.10010553.10010554</concept_id>
%  <concept_desc>Computer systems organization~Robotics</concept_desc>
%  <concept_significance>100</concept_significance>
% </concept>
% <concept>
%  <concept_id>10003033.10003083.10003095</concept_id>
%  <concept_desc>Networks~Network reliability</concept_desc>
%  <concept_significance>100</concept_significance>
% </concept>
%</ccs2012>
%\end{CCSXML}
%
%\ccsdesc[500]{Computer systems organization~Embedded systems}
%\ccsdesc[300]{Computer systems organization~Redundancy}
%\ccsdesc{Computer systems organization~Robotics}
%\ccsdesc[100]{Networks~Network reliability}
%
%%%
%%% Keywords. The author(s) should pick words that accurately describe
%%% the work being presented. Separate the keywords with commas.
%\keywords{datasets, neural networks, gaze detection, text tagging}
%%%
%% This command processes the author and affiliation and title
%% information and builds the first part of the formatted document.
\maketitle

\section{Introduction}
\label{sec:intro}
Recent years have seen renewed interest in using graphs for modelling, managing, querying and analysing data, particularly in scenarios involving diverse data, incomplete knowledge, multitudinous sources, and so forth. This interest stems from the growing realisation in various communities -- such as Databases~\cite{Bonifati}, Semantic Web~\cite{Hitzler2010}, Machine Learning~\cite{Hamilton}, and more recently Knowledge Graphs~\cite{HoganBCdMGKGNNN21} -- that graphs provide a flexible, lightweight and intuitive abstraction well-suited to many complex, diverse domains~\cite{AnglesABBFGLPPS18}.

Within these different communities, a variety of both abstract and concrete graph-based data models have been proposed.\footnote{By an \textit{abstract data model}, herein we refer to a structure used to represent data (e.g., the relational model). A \textit{concrete data model} further adds details important in practice, such as the types of terms allowed, syntaxes, etc. (e.g., SQL's data model).} Perhaps the simplest such model is that of a directed labelled graph, which is simply a set of triples, where each triple forms a directed labelled edge. This forms the basis of concrete data models, such as the Resource Description Framework~\cite{CyganiakWL14} proposed within the Semantic Web community. Such an abstraction is now also popular in the Machine Learning community, forming the basis of topics such as knowledge graph embeddings~\cite{Wang2017KGEmbedding}. It is also popular in the Database community, where such graphs are often simply called \textit{graph databases}, particularly in the more theoretical literature~\cite{Wood12}.

However, in practice, directed labelled graphs are sometimes considered \textit{too simple}. What if, for example, we want to add data that describe edges themselves, or graphs themselves? While more complex data can be modelled in directed labelled graphs using various forms of reification~\cite{HernandezHK15}, the result can often be verbose and unintuitive~\cite{AnglesABBFGLPPS18}. Hence a wide range of graph-based models have emerged down through the years~\cite{AnglesG08}. More recently, the (labelled) property graph model~\cite{Webber12,AnglesABBFGLPPS18} has gained significant popularity in the Database community, while models such as named graphs~\cite{HarrisS13} and RDF-star~\cite{Hartig17} have been proposed within the Semantic Web community as alternatives to reification. 

With several abstract and concrete graph models now available, the question becomes how to make these models \textit{interoperable}. How can we integrate data from both RDF and property graphs? How can we design a graph database engine that can seamlessly ingest, integrate and query data from any such model? How can we layer graph analytics or machine learning over such models? One possibility is to take the simplest model -- directed labelled graphs -- as our base and use reification~\cite{HernandezHK15} to represent more complex models, but as mentioned before, reification is too verbose. Another possibility is to take a more complex model -- say property graphs -- as our base~\cite{Hartig14,AnglesTT20}, but this would add complexity to higher levels when we think of graph queries, analytics, learning, etc.

Herein, we propose an intermediate solution. Specifically we propose \textit{one graph}: an abstract graph model that extends directed labelled graphs with edge ids~\cite{IlievskiGCDYRLL20,LassilaSBBBKKLST,VrgocRAAABHNRR21}. This model removes the need for reification when dealing with more complex models, and yet adds minimal complexity versus directed labelled graphs. 

We will first introduce existing graph models, and their strengths and weaknesses. We then introduce, motivate, and formally define the one graph model, and show how other graph models can be represented within it. We further discuss practical benefits of the model, and how it is currently being used. We conclude with some future directions regarding the model.

\section{Existing Graph Data Models}
\label{sec:existing}
One of the simplest models used for representing knowledge graphs is based on \textit{directed labelled graphs}, which consist of a set of nodes and edges such that every edge relates a pair of nodes. An edge is graphically represented as \gedge[arrin][0.6cm]{$a$}{$b$}{$c$}, where $a$ is called the source node, $b$ the edge label, and $c$ the target node. Such graphs are a staple in the theoretical literature on graph databases \cite{Baeza13}, and they form the basis of the RDF data model~\cite{CyganiakWL14}, where the source node, edge label and target node are called \textit{subject}, \textit{predicate} and \textit{object}, respectively. In the context of knowledge graphs, nodes are used to represent entities and edges represent binary relations. As an example, the edge \gedge[arrin][1.8cm]{Michelle Bachelet}{position held}{President of Chile}, tells us that Michelle Bachelet was (or is) the president of Chile.
%A popular concrete model based on \textit{directed edge-labelled graphs} is the RDF model~\cite{CyganiakWL14}. 

However, directed labelled graphs are sometimes \textit{too} simple. While they elegantly represent binary relations, they are cumbersome when representing higher arity relations. Take, for example, the two statements from the Wikidata knowledge graph~\cite{VrandecicK14} illustrated in Figure~\ref{fig:mb}. Both statements claim that Michelle Bachelet was a president of Chile, and both are associated with nested \textit{qualifiers} that provide additional information: in this case a start date, an end date, who replaced her, and whom she was replaced by. There are two statements, indicating two distinct periods when she held the position. Also the ids for objects (for example, \textsf{Q320} and \textsf{P39}) are shown; any positional element can have an id and be viewed as a node in the knowledge graph (for instance \texttt{start date}, identified by \texttt{P580} can be a source node of another statement). 

Representing statements like this in a directed labelled graph requires some form of \textit{reification} to decompose $n$-ary relations into binary relations~\cite{HernandezHK15}. For example, Figure~\ref{fig:delg} shows a graph where $e_1$ and $e_2$ are nodes representing $n$-ary relationships. The reification is given by the use of the edges labelled as \texttt{source}, \texttt{label} and \texttt{target}. For simplicity, we use human-readable nodes and labels, where in practice, a node \gnode{Sebastián Piñera} will rather be given as the identifier \gnode{Q306}, and an edge label \texttt{replaces} will rather be given as \texttt{P155}. While using reification is a valid solution, its main drawbacks are that: (i) it can easily become cumbersome and inefficient for querying; and (ii) it introduces semantics into graph data (requiring that an edge labelled \texttt{replaces} has a particular meaning for instance).

\newcommand{\wid}[1]{{\color{gray}~[#1]}} % format Wikidata IDs

\begin{figure}[tb]
\centering
{\sf
\begin{tabular}{l}
\large\textbf{Michelle Bachelet\wid{Q320}}\\[1ex]
\large\qquad position held\wid{P39} \quad President of Chile\wid{Q466956}\\
\qquad \qquad 
{\begin{tabular}{l@{\qquad~~}l}
start date \wid{P580} & 2014-03-11\\
end date \wid{P582} 	& 2018-03-11\\
replaces\wid{P155}	& Sebastián Piñera\wid{Q306}\\
replaced by\wid{P156} & Sebastián Piñera\wid{Q306}\\
\end{tabular}}\\
\\[-1ex]
\large\qquad position held\wid{P39} \quad President of Chile\wid{Q466956}\\
\qquad \qquad {\begin{tabular}{l@{\qquad~~}l}
start date \wid{P580} & 2006-03-11\\
end date \wid{P582} & 2010-03-11\\
replaces\wid{P155} & Ricardo Lagos\wid{Q331}\\
replaced by\wid{P156} & Sebastián Piñera\wid{Q306}\\
\end{tabular}}
\end{tabular}
}
\caption{Wikidata statement group for Michelle Bachelet \label{fig:mb}}
\end{figure}


%\begin{figure}[t]
%\centering
%\includegraphics[width=8.5cm]{./images/digraph.png}
%\caption{Directed edge-labelled graph reifying the statements of Figure~\ref{fig:mb} \label{fig:delg}}
%\end{figure}


\begin{figure}[t]
\setlength{\vgap}{0.45cm}
\setlength{\hgap}{2.5cm}
\centering
\begin{tikzpicture}
\node[iri] (e1) {$e_1$};

\node[iri,right=\hgap of e1] (pe1) {position held}
  edge[arrin] node[lab] {label} (e1);
  
\node[iri,above=\vgap of pe1] (se1) {Michelle Bachelet}
  edge[arrin] node[lab] {source} (e1);
  
\node[iri,below=\vgap of pe1] (oe1) {President of Chile}
  edge[arrin] node[lab] {target} (e1);
 
\node[iri,above=\vgap of se1] (sp) {Sebastian Piñera}
  edge[arrin] node[lab] {replaced by} (e1)
  edge[arrin,bend right=19] node[lab,pos=0.45] {replaces} (e1);
  
\node[iri,right=\hgap of pe1] (e2) {$e_2$}
  edge[arrout] node[lab] {label} (pe1)
  edge[arrout] node[lab] {source} (se1)
  edge[arrout] node[lab] {target} (oe1)
  edge[arrout] node[lab] {replaced by} (sp);
  
\node[iri,below=2.2\vgap of e1] (sd1) {2014-03-11}
  edge[arrin] node[lab] {start date} (e1);
\node[iri,above=2.2\vgap of e1] (ed1) {2018-03-11}
  edge[arrin] node[lab] {end date} (e1);

\node[iri,below=2.2\vgap of e2] (sd2) {2006-03-11}
  edge[arrin] node[lab] {start date} (e2);
\node[iri,above=2.2\vgap of e2] (ed2) {2010-03-11}
  edge[arrin] node[lab] {end date} (e2);
  
\node[iri,below=\vgap of oe1] (rl) {Ricardo Lagos}
  edge[arrin,bend right=17] node[lab] {replaces} (e2);
\end{tikzpicture}
\caption{Directed edge-labelled graph reifying the statements of Figure~\ref{fig:mb} \label{fig:delg}}
\end{figure}



To circumvent this issue, a number of graph models have been proposed to capture higher-arity relations more concisely, including property graphs~\cite{FrancisGGLLMPRS18} and RDF*~\cite{Hartig17,Hartig21}. However, both have limitations that render them incapable of modeling the statements shown in Figure~\ref{fig:mb} without resorting to reification~\cite{HoganRRS19}. On the one hand, property graphs allow labels and property--value pairs to be associated with both nodes and edges. For example, the statements of Figure~\ref{fig:mb} can be represented as the property graph in Figure \ref{fig:pg}. 
Though more concise than reification, labels, properties and values are considered to be simple strings, which are disjoint with nodes; for example, \textsf{"Ricardo Lagos"} is neither a node nor a pointer to a node, but a string, which would complicate, for example, querying for the parties of presidents that Michelle Bachelet replaced. 

\begin{figure}
\setlength{\vgap}{1.5cm}
\setlength{\hgap}{1cm}
\centering
%\begin{center}
\vspace{2pt}
\setlength{\vgap}{1.5cm}
\setlength{\hgap}{1.5cm}
\begin{tikzpicture}
  %%% node n1 
  \node[nrect] (n1) {
   \alt{
       \uri{name} & =\,\uri{"Michelle Bachelet"} } };
  %%% label of n1
  \node[rt] (ln1) at (n1.north) {$n_1$ : \uri{human}};
  
  %%% node n2
  \node[nrect,right=\hgap of n1] (n2) {
   \alt{
      \uri{name} & =\,\uri{"President of Chile"} } };
  %%% label of n2
  \node[rt] (ln2) at (n2.north) {$n_2$ : \uri{public office}};

  %%% edge e1
  \draw[arrout,pos=0.5,bend left=25] (ln1) to node[rte,yshift=-0.1cm] (le1)
   {$e_1$ $:$ position held}
  (ln2);
  %%% atributes e1
  \node[erect,anchor=south,yshift=-0.5ex] at (le1.north) (e1) {
     \alt{ \uri{start date} & =\,\uri{"11 March 2014"}\\[-0.7em]
           \uri{end date} & =\,\uri{"11 March 2018"}\\[-0.7em]
           \uri{replaces} & =\,\uri{"Sebastián Piñera"}\\[-0.7em]
           \uri{replaced by} & =\,\uri{"Sebastián Piñera"} } };
  
  %%% edge e2
  \draw[arrout,pos=0.5,bend right=25] (n1) to node[rte,yshift=-0.1cm] (le2) {$e_2$ $:$ position held} (n2);
  %%% atributes e2  
  \node[erect,anchor=north,yshift=0.5ex] (e2) at (le2.south) {
     \alt{ \uri{start date} & =\,\uri{"11 March 2006"}\\[-0.7em]
           \uri{end date} & =\,\uri{"11 March 2010"}\\[-0.7em]
           \uri{replaces} & =\,\uri{"Ricardo Lagos"}\\[-0.7em]
           \uri{replaced by} & =\,\uri{"Sebastián Piñera"}\\ } };
\end{tikzpicture}
\vspace{2pt}
%\end{center}
\caption{Property graph for the statements of Figure~\ref{fig:mb}, with some additional data\label{fig:pg}}
\end{figure}




On the other hand, RDF* allows an edge to be a node. For example, the first statement of Figure~\ref{fig:mb} can be represented as follows in RDF* as shown in Figure \ref{fig:rdf*}. The node representing the edge is called a quoted triple~\cite{Hartig21}.
However, we can only represent one of the statements (without reification), as we can only have one distinct node per edge; if we add the qualifiers for both statements, then we would not know which start date pairs with which end date, for example. A proposed workaround involves adding intermediate nodes to denote different \textit{occurrences} of quoted triples, but this requires a reserved term~\cite{Hartig21}.
%Finally, RDF datasets model multiple named graphs, where we could define a graph with a single edge and define qualifiers on that graph's name~\cite{HernandezHK15}; however, named graphs are intended for tracking the source of a graph.


\begin{figure}[tb]
\setlength{\vgap}{0.8cm}
\setlength{\hgap}{1.9cm}
\centering
\begin{tikzpicture}
\node[iri,anchor=center,minimum width=6.3cm,minimum height=0.7cm,dashed] (e1) {};
\node[iri,right=1ex of e1.west,anchor=west] (sp1) {Michelle Bachelet};
\node[iri,right=\hgap of sp1] (ch1) {President of Chile}
  edge[arrin] node[iri,draw=none] {position held} (sp1);
  
\node[iri,below=\vgap of e1] (mb) {Sebastián Piñera}
  edge[arrin,bend right=40] node[lab,xshift=1ex] {replaces} (e1)
  edge[arrin,bend left=40] node[lab,xshift=-1ex] {replaced by} (e1);

\node[iri,left=\hgap of mb] (ttt) {2018-03-11}
  edge[arrin] node[lab] {start date} (e1);

\node[iri,right=\hgap of mb] (ttf) {2014-03-11}
  edge[arrin] node[lab] {end date} (e1);
\end{tikzpicture}
\caption{RDF* graph for the first statement of Figure~\ref{fig:mb} \label{fig:rdf*}}
\end{figure}



\section{One Graph (1G)}
\label{sec:data}
As illustrated in the previous section, one of the key features we need to model complex statements, such as multiple presidencies of a particular person, is the ability to refer to the entire statement repeatedly. More precisely, we need to be able to use a graph edge as a source of another node. We capture this concept by introducing an data model called \Data, which   assignin ids to edges in order to capture higher-arity relations within graphs~\cite{HernandezHK15,IlievskiGCDYRLL20,LassilaSBBBKKLST}. Formally, assume a universe $\objs$ of objects (ids, strings, numbers, IRIs, etc.). We define \datas as follows:

\begin{definition}
A \emph{one graph} $G = (O,\gamma)$ consists of a finite set of objects $O\subseteq \objs$ and a partial mapping $\gamma : O \rightarrow O \times O \times O$. 
\end{definition}

Intuitively, $O$ is the set of objects that appear in our graph database, and $\gamma$ models edges between objects. If $\gamma(e)=(n_1,t,n_2)$, this states that the edge $(n_1,t,n_2)$ has id $e$, type $t$, and links the source node $n_1$ to the target node $n_2$.\footnote{Herein, we say ``\textit{edge type}'' rather than ``\textit{edge label}'' to highlight that the type forms part of the edge, rather than being an annotation on the edge, as in property graphs.} We can analogously define our model as a relation:
\[\textsc{Connections}(\textsf{source},\textsf{type},\textsf{target},\underline{\textsf{eid}}) \]
where \underline{\textsf{eid}} (edge id) is a primary key of the relation. 


We stated that \data model allows us to capture higher-arity relations more directly. So how do we represent the WikiData statements from  Figure \ref{fig:mb}? One possible representation is given in  Figure \ref{fig:dg}. We only show edge ids as needed (all edges have ids). 

\begin{figure}[t]
\setlength{\vgap}{0.8cm}
\setlength{\hgap}{1.6cm}
\centering
\begin{tikzpicture}
\node[iri,anchor=west,yshift=1ex] (sp1) {Michelle Bachelet};
\node[iri,right=\hgap of sp1] (ch1) {President of Chile}
  edge[arrin,bend left=40] node[enode] (e1) {$e_1:$ position held} (sp1)
  edge[arrin,bend right=40] node[enode] (e2) {$e_2:$  position held} (sp1);

\node[iri,below=\vgap of e1] (ttt) {2014-03-11}
  edge[arrin] node[lab] {start date} (e1);
  
\node[iri,left=0.5\hgap of sp1] (mb) {Sebastián Piñera}
  edge[arrin,bend right=30] node[lab,xshift=1ex] {replaces} (e1.210)
  edge[arrin,bend right=10] node[lab,xshift=-1ex] {replaced by} (e1.190)
  edge[arrin,bend left=20] node[lab,xshift=-1ex] {replaced by} (e2.170);

\node[iri,right=0.5\hgap of ttt] (ttf) {2018-03-11}
  edge[arrin] node[lab,xshift=1ex] {end date} (e1);

\node[iri,above=\vgap of e2] (ttt2) {2006-03-11}
  edge[arrin] node[lab] {start date} (e2);
  
\node[iri,left=0.5\hgap of ttt2] (rl) {Ricardo Lagos}
  edge[arrin] node[lab] {replaces} (e2);

\node[iri,right=0.5\hgap of ttt2] (ttf2) {2010-03-11}
  edge[arrin] node[lab,xshift=1ex] {end date} (e2);

%\node[iri,anchor=center,minimum width=6.6cm,minimum height=0.7cm,below=mb] (e1) {};

%\node[anchor=east,left=0.1ex of e1.east] (e1i) {$e_1$};
\end{tikzpicture}
\caption{Domain graph for Figure~\ref{fig:mb} \label{fig:dg}}
\end{figure}



One could argue that a more natural representation would be to have something similar to the property graph of Figure \ref{fig:pg}, but now with attributes and values that can again be graph objects. After all, the representation presented here is somewhat similar to the reification we gave before, but with better properties. The solution to this is what we call the property-\Data, a full data model allowing edges as  nodes, multiple (referenceable) labels per node, as well property/value pairs where each art can again be a graph node. Formally, we define:

%\ah{I am not sure if see the distinction between edge type/edge label/predicate, source/subject, target/object, connection/edge/relation? If there is a distinction, we should make it clear. If there is no distinction, I think we should stick to one preferred nomenclature for the paper (even if diverse nomenclature is used elsewhere or in the implementation). Personally I prefer edge label and edge as they are more widely accepted and more traditional. I am indifferent about source/subject and target/object, but if we go for edge label/type rather than predicate, then source and target might be more consistent. Nodes denote entities and edges encode relations(hips) between entities.}

\begin{definition}
 A \emph{property \data} is defined as a tuple $G=(O,\gamma,\lab,\prop)$, where:
\begin{itemize}
%    \item $O \subseteq \objs$, $L \subseteq \labels$, $P \subseteq \cprop$ and $V \subseteq \values$ are finite sets of objects, labels, properties and values, respectively;
    \item $(O,\gamma)$ is a \data;
    \item $\lab : O \rightarrow 2^O$ is a function
    %the labelling
    assigning a finite set of labels to an object; and
    \item $\prop : O \times O \rightarrow O$ is a partial function assigning a value to a certain property of an object. 
    %, to its value.
\end{itemize}
Moreover, we assume that for each object $o \in O$, there exists a finite number of properties $p \in O$ such that $\prop(o,p)$ is defined.
\end{definition}

The statements of Figure \ref{fig:mb} can now be represented as a property-\data almost identical to the property graph from Figure \ref{fig:pg}, but now having each label, property, and its value to be a graph object. More precisely, the property \texttt{start date} of the edge $e_1$ would again be a graph object, as would the value of the property \texttt{replaces}; i.e. \texttt{Sebasti\'an Pi\~nera}. Similarly, the label \texttt{human} of the node $n_1$ can also be an object in a property-\data\footnote{If needed, one can also define a special set of literals disjoint from $\objs$, and specify that labels (resp. properties and their values) can be either literals or referenceable objects}. %Property-\datas then naturally capture property graphs.

 The relational representation of property \data then adds two new relations alongside \textsc{Connections}:
\begin{center}
$\textsc{Labels}(\textsf{object},\textsf{label})$,\\
$\textsc{Properties}(\underline{\textsf{object},\textsf{property}},\textsf{value}),$
\end{center}
%
where \underline{\textsf{object},\,\textsf{property}} is a primary key of the second relation, with the first relation allowing multiple labels per object.
%Relation $\textsc{Labels}$ allows multiple labels per object, while relation $\textsc{Properties}$ allows multiple values per object and property.


\subsection{\Data and other graph models}\label{ssec:whydg}

The \data data model naturally subsumes the RDF graph model, as well as RDF*. To show how RDF is modeled in \datas, consider the following edge, claiming that Michelle Bachelet was the president of Chile.

\medskip
\begin{center}
\gedge[arrin][2.5cm]{Michelle Bachelet}{position held}{President of Chile}
\end{center}
\medskip

\noindent
We can encode this triple in a \data by storing the tuple \textsf{(Michelle Bachelet, position held, President of Chile, e)} in the \textsf{DomainGraph} relation, where \textsf{e} denotes a unique (potentially auto-generated) edge id, or equivalently stating that:
$$\gamma(\textsf{e}) = \textsf{(Michelle Bachelet, position held, President of Chile)}.$$
Here the edge id can be automatically generated. One can thus automatically load an RDF dataset into \Data, by assigning a new edge id to each triple. The id of the edge itself is not needed in the RDF data model, but it can be used for modelling RDF-star (RDF*) graphs. On the other hand, the property-\Data model naturally subsumes property graphs, where each property graph can be taken verbatim to be a property-\Data.

%\Datas are inspired by \textit{named graphs} in RDF/SPARQL (where named graphs have one triple/edge each). Both domain graphs and named graphs can be represented as quads. Named graphs were proposed to represent multiple RDF graphs for publishing and querying. SPARQL thus uses syntax -- such as \textsf{FROM}, \textsf{FROM NAMED}, \textsf{GRAPH} -- that is unintuitive when querying singleton named graphs representing higher-arity relations. Moreover, SPARQL does not support querying paths that span named graphs; in order to support path queries over singleton named graphs, all edges would need to be duplicated (virtually or physically) into a single graph~\cite{HernandezHK15}. Named graphs (with multiple edges) could be supported in domain graphs using a reserved term \textsf{graph}, and edges of the form $\gamma(e_3) = (e_1,\textsf{graph},g_1)$, $\gamma(e_4) =(e_2,\textsf{graph},g_1)$; optionally, \textit{named domain graphs} could be considered in the future to support multiple domain graphs with quins.


The idea of assigning ids to edges/triples for similar purposes as described here is a natural one, and not new to this work. \citet{HernandezHK15} explored using singleton named graphs in order to represent Wikidata qualifiers, placing one triple in each named graph, such that the name acts as an id for the triple. In parallel with our work, recently a data model analogous to domain graphs has been independently proposed for use in Amazon Neptune, which the authors call 1G~\cite{LassilaSBBBKKLST}. Their proposal does not discuss a formal definition for the model, nor a query language, storage and indexing, implementation, etc., but the reasoning and justification that they put forward for the model is similar to ours. To the best of our knowledge, our work is the first to describe a query language, storage and indexing schemes, query planner -- and ultimately a fully-fledged graph database engine -- built specifically for this model. Furthermore, with property domain graphs, we support annotation external to the graph, which we believe to be a useful extension that enables better compatibility with property graphs.

Table~\ref{tab:gmodel} summarizes the features that are directly supported by the respective graph models themselves without requiring \textit{reserved terms}, which would include, for example, \textsf{source}, \textsf{label} and \textsf{target} in Figure~\ref{fig:delg} (all features except \textit{External annotation} can be supported in all models with reserved vocabulary). Reserved terms can add indirection to modeling (e.g., reification~\cite{HernandezHK15}), and can clutter the data, necessitating more tuples or higher-arity tuples to store, leading to more joins and/or index permutations. The features are then defined as follows, considering directed (labeled) edges:

\begin{itemize}
\item \textit{Edge type/label}: assign a type or label to an edge.
\item \textit{Node label}: assign labels to nodes.
\item \textit{Label as node}: label can be a node.
\item \textit{Edge annotation}: assign property--value pairs to an edge.
\item \textit{Node annotation}: assign property--value pairs to a node.
\item \textit{External annotation}: nodes/edges can be annotated without adding new nodes or edges.
\item \textit{Annotation name as node}: the property part of an annotation can be a node.
\item \textit{Annotation value as node}: the value part of an annotation can be a node.
\item \textit{Edge as node}: an edge can be referenced as a node (this allows edges to be connected to nodes of the graph).
\item \textit{Edge as nodes}: a single unique edge can be referenced as multiple nodes.
\item \textit{Nested edge nodes}: an edge involving an edge node can itself be referenced as a node, and so on, recursively.
\item \textit{Graph as node}: a graph can be referenced as a node.
\end{itemize}

Some \nmark's in Table~\ref{tab:gmodel} are more benign than others; for example, \textit{Node label} requires a reserved term (e.g., \textsf{rdf:type}), but no extra tuples; on the other hand, \textit{Edge as node} requires reification, using at least one extra tuple, and at least one reserved term.

Wikidata requires \textit{Edge as nodes} for cases as shown in Figure~\ref{fig:mb} (since values such as \textsf{Ricardo Lagos} are themselves nodes). Only named graphs, \datas and property \datas can model such examples without reserved terms. Comparing named graphs and \datas, the latter sacrifices the ``\textit{Graph as node}''. Such a feature is not needed by Wikidata, but could be added in future versions. Property \datas further support external annotation, and thus better compatibility with legacy property graphs.

%In summary, the \data model and the extended property \data model supported by MillenniumDB allow the user to treat their graph as a property graph, an RDF or RDF* graph, or a mix of both, akin to how Wikidata uses qualifiers on edges (like property graphs) whose properties and values can also be nodes/objects in the graph (like in RDF). Like RDF* or property graphs, domain graphs support nesting, where edges about edges are themselves recursively edges with ids; unlike RDF*, but like property graphs, the same edge can be described in multiple ways using different edge ids. Unlike named graphs, domain graphs can be indexed with fewer index permutations, and are designed from scratch for the purpose of modeling higher-arity relations.

\renewcommand{\nmark}{}

\begin{table}
\caption{The features supported by graph models without reserved terms (NG = Named Graphs, PG = Property Graphs, 1G = One Graphs, P1G = Property One Graphs) \label{tab:gmodel}}
\begin{tabular}{ccccccc}
\toprule
 & RDF & RDF* & NG & PG & 1G & P1G \\
\toprule
\textit{Edge type/label} & \ymark & \ymark & \ymark & \ymark & \ymark & \ymark \\
\textit{Node label} & \nmark & \nmark & \nmark & \ymark & \nmark & \ymark \\
\textit{Label as node} & \nmark & \nmark & \nmark & \nmark & \nmark & \ymark \\
\textit{Edge annotation} & \nmark & \ymark & \ymark & \ymark & \ymark & \ymark \\
\textit{Node annotation} & \ymark & \ymark & \ymark & \ymark & \ymark & \ymark \\
\textit{External annotation} & \nmark & \nmark & \nmark & \ymark & \nmark & \ymark \\
\textit{Annotation name as node} & \nmark & \nmark & \nmark & \nmark & \nmark & \ymark \\
\textit{Annotation value as node} & \nmark & \nmark & \nmark & \nmark & \nmark & \ymark \\
\textit{Edge as node} & \nmark & \ymark & \ymark & \nmark & \ymark & \ymark \\
\textit{Edge as nodes} & \nmark & \nmark & \ymark & \nmark & \ymark & \ymark \\
\textit{Nested edge nodes} & \nmark & \ymark & \ymark & \nmark & \ymark & \ymark \\
\textit{Graph as node} & \nmark & \nmark & \ymark & \nmark & \nmark & \nmark \\
\bottomrule
\end{tabular}
\end{table}




\section{Conclusions and looking ahead}
\label{sec:concl}
To reuse:

Domain graphs adopt the natural idea of adding edge ids to directed labeled edges in order to concisely model higher-arity relations in graphs, as needed in Wikidata, without the need for reserved vocabulary or reification. They can naturally represent popular graph models, such as RDF and property graphs, and allow for combining the features of both models in a novel way. While the idea of using edge ids as a hook for modeling higher-arity relations in graphs is far from new (see, e.g.,~\cite{HernandezHK15,IlievskiGCDYRLL20,LassilaSBBBKKLST}), it is an idea that is garnering increased attention as a more flexible and concise alternative to reification.
%, RDF*, property graphs, etc.
To the best of our knowledge, our work is the first to propose a formal data model that incorporates edge ids, a query language that can take advantage of them, and a fully-fledged graph database engine that supports them by design. We also propose to optionally allow (external) annotations on top of the graph structure, thus facilitating better compatibility with property graphs, whereby labels and property--values can be added to graph objects without adding new nodes and edges to the graph~itself.




%% The next two lines define the bibliography style to be used, and
%% the bibliography file.

\begin{acks}
Hogan, Rojas and Vrgo\v{c} are supported by ANID -- Millennium Science Initiative Program -- Code ICN17\_002. Vrgo\v{c} is also supported by Fondo Puente UC.
\end{acks}


\bibliographystyle{ACM-Reference-Format}

\newpage
%UNCOMMENT IF FULL BIB IS DESIRED
% \bibliography{biblio}
\bibliography{nourlbiblio}

%%
%% If your work has an appendix, this is the place to put it.

\end{document}
\endinput
%%
%% End of file `sample-sigconf.tex'.
